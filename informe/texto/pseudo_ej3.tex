\subsection{Pseudocodigo}

Para este ejercicio manejamos una cola q en la cual tenemos todos los procesos ready.




\begin{algorithm}[H]
\caption{SchedRR(Vector<int> argn)}
\label{pseudo:load}
\begin{algorithmic}

\STATE cantCores = argn[0]
\STATE cpuQuantum = Vector<int> de tamaño cantCores y cero en todas las posiciones.
\STATE quantumActual = Vector<int> de tamaño cantCores y cero en todas las posiciones.
\FOR{($i=0$ hasta $cantCores$)}
	\STATE quantumActual[i] = argn[i+2]
	\STATE cpuQuantum[i] = argn[i+2]	
\ENDFOR

\end{algorithmic}
\end{algorithm}




\begin{algorithm}[H]
\caption{load(pid)}
\label{pseudo:load}
\begin{algorithmic}

\STATE q.push(pid)

\end{algorithmic}
\end{algorithm}

\begin{algorithm}[H]
\caption{unblock(pid)}
\label{pseudo:unblock}
\begin{algorithmic}

\STATE q.push(pid)

\end{algorithmic}
\end{algorithm}


\begin{algorithm}[H]
\caption{tick( cpu , Motivo m)}
\label{pseudo:tick}
\begin{algorithmic}

 \IF{(m == EXIT o m == BLOCK)}
	\IF{(q.vacia())}
		\STATE Devolver run\_idle\_task(cpu)
	\ELSE
		\STATE quantumActual[cpu] = cpuQuantum[cpu]
		\STATE prox = q.front()
		\STATE q.pop()
		\STATE Devolver prox		
	\ENDIF
\ELSE
	\IF{(current\_pid(cpu) == IDLE\_TASK)}
		\IF{(q.vacia())}
			\STATE Devolver devolver run\_idle\_task(cpu)
		\ELSE
			\STATE Devolver proximo\_proceso(cpu)
		\ENDIF
	\ELSE
		\STATE quantumActual[cpu] = quantumActual[cpu] - 1
		\IF{(quantumActual[cpu] == 0)}
			\IF{(q.vacia())}
				\STATE quantumActual[cpu] = cpuQuantum[cpu]
				\STATE Devolver current\_pid(cpu)
			\ELSE
				\STATE Devolver proximo\_proceso(cpu)
			\ENDIF
		\ELSE
			\STATE Devolver current\_pid(cpu)
		\ENDIF
	\ENDIF
	
 \ENDIF

\end{algorithmic}
\end{algorithm}


\begin{algorithm}[H]
\caption{proximo\_proceso(cpu)}
\label{pseudo:proximoproceso}
\begin{algorithmic}

\STATE quantumActual[cpu] = cpuQuantum[cpu]
\STATE prox = q.front()
\STATE q.pop()
\IF{(current\_pid(cpu) != -1)}
	\STATE cola.push(current\_pid(cpu))
\ENDIF
\STATE Devolver prox

\end{algorithmic}
\end{algorithm}


\begin{algorithm}[H]
\caption{run\_idle\_task(cpu)}
\label{pseudo:runidletask}
\begin{algorithmic}

\STATE quantumActual[cpu] = cpuQuantum[cpu]
\STATE Devolver IDLE\_TASK

\end{algorithmic}
\end{algorithm}


\subsection{Explicación de la implementacion}

Para implementar el scheduler Round Robin utilizamos una cola de procesos \textit{ready}. Cuando un núcleo del procesador queda libre \textit{(ya sea por que el proceso ejecutándose termino su quantum, se bloqueo o termino su ejecución)} hacemos que el próximo proceso a ejecutarse en ese núcleo sea el primer proceso en la cola de procesos \textit{ready} \textit{(en caso de no haber ninguno, el procesador queda idle)}.

Cuando un proceso se desbloquea es colocado al final de la cola de procesos \textit{ready} a la espera de que algún núcleo del procesador quede libre para ejecutarse.

Si un núcleo esta idle \textit{(es decir, ejecutando la tarea idle)} y no hay ningún proceso \textit{ready} para ejecutar, simplemente se queda a la espera de que algún proceso este \textit{ready}.
