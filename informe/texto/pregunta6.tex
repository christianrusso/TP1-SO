\section{Pregunta 6}

\subsection{Enunciado del Problema:}

Programar un tipo de tarea TaskBatch que reciba dos parametros: total cpu y cant bloqueos. Una tarea de este tipo debera realizar cant bloqueos llamadas bloqueantes, en momentos elegidos pseudoaleatoriamente. En cada tal ocasion, la tarea debera permanecer bloqueada durante exactamente un (1) ciclo de reloj. El tiempo de CPU total que utilice una tarea TaskBatch deberia ser de total cpu ciclos de reloj (incluyendo el tiempo utilizado para lanzar las llamadas bloqueantes; no asi el tiempo en que la tarea permanezca bloqueada).
\subsection{Pseudocodigo}

\begin{algorithm}[H]
\caption{TaskBatch(pid,  params)}
\label{pseudo:TaskBatch}
\begin{algorithmic}

\STATE ciclosTotales = params[0]
\STATE entradasBloqueantes = params[1]
\STATE vector<int> ocupado(ciclosTotales,0)
\FOR{$i=0$ hasta $entradasBloqueantes$}
    \STATE momento = rand() \% ciclosTotales 
    \WHILE{ ocupado[momento]}
	\STATE momento = rand() \% ciclosTotales
    \ENDWHILE
    
    \STATE ocupado[momento] = 1
\ENDFOR

\FOR{$i=0$ hasta $ciclosTotales$}
\IF{(ocupado[i])}
	\STATE uso\_IO(pid, 1)
\ELSE
	\STATE uso\_CPU(pid, 1)
\ENDIF
\ENDFOR
\end{algorithmic}
\end{algorithm}

\subsection{Explicación de la implementacion}

Se crea un vector de tamaño $ciclosTotales$ (es la cantidad total de ciclos de uso de CPU de la tarea) inicializado en cero. Lo interpretaremos como un vector de booleanos, donde si la posición i esta en 1, es porque en el momento i se hace una llamada bloqueante y cero en caso de que sea un ciclo solo de uso de CPU. Al principio en todas los momentos se esta usando solo CPU.

Luego se itera $entradasBloqueantes$ (es la cantidad de llamadas bloqueantes en la tarea) veces y se elije un momento al azar entre 0 y $ciclosTotales$. Si ese momento ya estaba ocupado ($ocupado[momento] = TRUE$) entonces se vuelve a elegir al azar para tomar otro momento. Luego se setea en uno indicando que habrá una llamada bloqueante en ese momento.

Luego se recorre el vector y se ejecuta la tarea. Si el $ocupado[momento] = TRUE$ entonces se ejecuta un ciclo de uso de CPU y una llamada bloqueante, sino se ejecuta un solo ciclo de CPU.